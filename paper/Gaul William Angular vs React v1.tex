\documentclass[12pt,letterpaper]{article}

% Import settings, which are stored in a separate file for convenience
% Syntax highlighting colors
\definecolor{lightgray}{rgb}{0.9, 0.9, 0.9}
\definecolor{darkgray}{rgb}{0.4, 0.4, 0.4}
\definecolor{purple}{rgb}{0.65, 0.12, 0.82}

\lstdefinelanguage{JavaScript}{
	keywords={break, case, catch, continue, debugger, default, delete, do, else, false, finally, for, function, if, in, instanceof, new, null, return, switch, this, throw, true, try, typeof, var, void, while, with},
	morecomment=[l]{//},
	morecomment=[s]{/*}{*/},
	morestring=[b]',
	morestring=[b]",
	ndkeywords={class, export, boolean, throw, implements, import, this},
	sensitive=true
}

\lstdefinestyle{js}{
	language=JavaScript,
	keywordstyle=\color{blue}\bfseries,
	ndkeywordstyle=\color{darkgray}\bfseries,
	identifierstyle=\color{black},
	commentstyle=\color{purple}\ttfamily,
	stringstyle=\color{red}\ttfamily,
	tabsize=2
}

\lstdefinestyle{html}{
	language=HTML,
	keywordstyle=\color{blue}\bfseries,
	ndkeywordstyle=\color{darkgray}\bfseries,
	identifierstyle=\color{black},
	commentstyle=\color{purple}\ttfamily,
	stringstyle=\color{red}\ttfamily,
	tabsize=4
}


% Set global defaults for code listings
\lstset{
	backgroundcolor=\color{lightgray},
	basicstyle=\singlespacing\footnotesize\ttfamily,
	breaklines=true,
	captionpos=b,
	extendedchars=true,
	showspaces=false,
	showstringspaces=false,
	showtabs=false,
	upquote=true
}


% Sensible defaults for spacing and line breaks
\def\UrlBreaks{\do-\do_}
\titlespacing*{\section}{0pt}{0pt}{0pt}
\titlespacing*{\subsection}{0pt}{0pt}{0pt}
\doublespacing
\makeatletter


% Customize title
\renewcommand{\maketitle}{
	\begin{flushright}
		\@author\\\@date
	\end{flushright}
	\begin{center}
		{\LARGE\@title}
	\end{center}
}


% Set title
\title{\textbf{Angular vs. React}}
\author{William Gaul\\{ICS 419: Prof. Streveler}}
\date{\today}


\begin{document}
\maketitle

\begin{singlespace*}
\begin{abstract}
	The modern web is inherently dynamic. As the Internet has evolved -- as JavaScript plays an increasingly ubiquitous role and applications grow in size and complexity -- developers have turned ever more toward \emph{web frameworks} to make sense of the chaos. Many companies now favor applicants who have experience working with these frameworks, and so a good understanding of their usage and purpose is crucial. Why do they exist? What problems do they purport to solve, if any? How will they benefit developers and end users alike?

	This paper answers these questions by examining two very popular web frameworks in more detail: Angular and React. The two frameworks are first compared against pure JavaScript and each other by way of a simple dynamic web page. The objectives and contrasting designs of both frameworks are then explored in the context of this example. Finally, a goal for each framework is proposed and its impact on web development considered.
\end{abstract}
\end{singlespace*}


\section{The Web, Then and Now}
\vspace{-12pt}

\begin{quote}
	\singlespacing
	\emph{In which a number of problems caused by the evolution of the web, and motivating the creation of web frameworks, is discussed.}
\end{quote}

The web then was, as many might now say, boring.

Tim Berners-Lee, the father of the web, introduced HTML in 1989 as an easy way to produce and disseminate information across the fledgling scientific community called the Internet. HTML encapsulated the notion of documents described by something called hypertext: rich content that contains \emph{hyperlinks} pointing to other documents. From the start, HTML was ``intended for simple document structure, not for handling today's varied and complex information needs'' \cite[p.~9]{Sklar:2012} -- that is, static web pages \cite{Schlensker:2014}.

By the mid-1990s, the web began to gain widespread popularity, causing companies and individuals alike to focus on developing a web presence. This, coupled with the lack of authoritative standards, led to an explosion of questionable online design practices \footnote{As an example, HTML tables were once used for entire page layouts although they are only meant for tabulating data, simply because they are easier to work with.}. Problems with interoperability were only exacerbated by the introduction of CSS and JavaScript. Although these new technologies were meant to simplify development by separating aesthetic and functional concerns from HTML content, they paved the way for greater complexity through manipulation of HTML elements \cite{Sklar:2012}.

The desire for more data-driven applications ushered in an era commonly known as ``Web 2.0,'' characterized by Dynamic HTML (DHTML) and rudimentary animated interactivity. At around this time, developers also began to feel the pangs of cross-platform development, as major browser vendors each tried to implement their own version of the HTML specification. These difficulties would only mount as Internet-capable mobile devices made their debut, forcing many designers to adapt existing sites for mobile-friendly consumption. \cite{Schlensker:2014}.

Fast-forward to the present: static, document-based web pages have largely been replaced by dynamic single-page applications (SPAs). JavaScript is a mature language with a strong community, and coupled with the recent release of CSS3, forms the cornerstone of the HTML5 specification that ``attempts to address \ldots the needs of modern Web design and the application-based future of the Web.'' \cite[p.~14]{Sklar:2012}. Development is no longer simply about cultivating a web presence -- rather, there is a distinct culture of \emph{integration} with existing services such as Twitter, Disqus, and GitHub.

What problems has this evolution caused? As applications continue to grow in complexity and scale, their architectures become increasingly difficult to reason about. With such weak standardization across the Internet, a great deal of code and logic is needlessly reimplemented \cite{Schlensker:2014}. Perhaps most troubling, however, is that HTML is \emph{still not designed for dynamic applications}, causing an ``impedance mismatch'' between what developers want to do and what web platforms are capable of supporting \cite{Angular:Docs, Hunt:2014}.

Fortunately, a solution exists: web frameworks. Like the web itself, frameworks have evolved over time; many modern frameworks had their start as JavaScript libraries that focused on abstracting away platform-specific details \footnote{jQuery, one such library that is still very popular, is an excellent abstraction for event listeners and Ajax calls. Another library, Moment.js, handles parsing and manipulating dates.}. Frameworks are also platform abstractions, but decidedly more comprehensive and opinionated. They improve upon pure JavaScript by not only facilitating clean, DRY \footnote{An acronym for Don't Repeat Yourself, referring to the elimination of repetitive logic in a program.} code, but also by providing extensible architectures to accommodate the diverse requirements of a modern application \cite{Schlensker:2014}.


\section{Hello Dynamic World}
\vspace{-12pt}

\begin{quote}
	\singlespacing
	\emph{In which the author uses a simple dynamic web page to investigate the designs of Angular and React.}
\end{quote}

To gain a better understanding of how frameworks improve upon pure JavaScript, it is necessary to consider a simple example -- in this case, the web page shown in Figure \ref{fig:Example}.

\begin{SCfigure}[][h]
	\centering
	\caption{A simple example of a dynamic web element: the header text changes based on user input. Code for this example in pure JavaScript, as well as using the Angular and React frameworks, can be found in Appendix \ref{app:code}.}
	\fbox{\includegraphics[width=0.5\textwidth]{hello-world.png}}
	\label{fig:Example}
\end{SCfigure}

Syncing the contents of an \texttt{<h1>} element to an input is fairly straightforward with pure JavaScript: attach an \emph{event listener} to the input, and update the header's \texttt{innerHTML} accordingly whenever an event is fired. The problem with this approach is that it does not scale well. Imagine having to handle the updates of several thousand UI elements, a typical number for a web application as complex as Facebook. This pattern would have to be duplicated thousands of times, and the codebase would quickly become unmanageable.

In contrast, the Angular code contains no such low-level event handling. Rather, a JavaScript variable is introduced into the application's scope and bound to both the input and header \footnote{The variable, \texttt{\$scope.subject}, is specified as the input's \emph{model} via the \texttt{ng-model} attribute and the header's content via an inline template.}. This concept is known as \emph{two-way data binding} because updates may flow in both directions; when a user modifies the input its underlying model (the variable) is updated, and similarly when the variable is modified via JavaScript its new value will be reflected in the UI \cite{Angular:Docs}. The resulting code is highly declarative in nature, and moreover very terse, as nearly all of the low-level logic is handled by Angular behind the scenes.

React takes a different tack by removing HTML from the picture entirely. In its place are React components that encapsulate state and business logic, and that may be nested to construct complex applications. Updates are handled by describing the UI as it should appear at any discrete moment in time within the \texttt{render/0} function -- whenever the state of a component changes, it is automatically ``re-rendered'' \footnote{Here the markup for the input depends upon \texttt{this.state.subject}. Whenever the input is changed, \texttt{this.setState/1} is triggered with the new value, and the UI is rendered again.}. Because events from the user must still be handled manually, this is known as \emph{one-way data binding} \cite{Hunt:2014}.

One-way versus two-way data binding may seem like a simple distinction, but it has profound implications for how an application is designed, developed, and even reasoned about. A closer look at the code will help reveal how this design choice has guided the goals and objectives of both frameworks.

First, consider that one of the major tenets in the so-called ``Zen of Angular'' is that a framework should ``[guide] developers through the entire journey of building an app'' \cite[p.~1]{Angular:Docs}. Angular does this by extending HTML to provide facilities for everything from form validation to reusable components (known as \emph{directives} in Angular) to testing \cite{Green:2013}. In other words, Angular tries to accommodate the entire development lifecycle, with the goal of being a complete solution for building web applications. Two-way data binding is thus a perfect match because it abstracts away as much as possible: the provided example includes no application logic, and nearly all structure is expressed in HTML.

The goal of React can be summed up in one quote \cite{Hunt:2014}:

\begin{quote}
	\singlespacing
	``We should do our utmost to shorten the conceptual gap between the static program and the dynamic process.''

	\raggedleft --- Edsger Dijkstra
\end{quote}

\noindent Whereas Angular is ``not good for intensive and tricky DOM manipulation'' \cite[p.~2]{Angular:Docs}, React excels at keeping a UI in sync with its underlying state. Unlike Angular, which tries to satisfy all development needs, React has a focused objective: encourage ``the creation of reusable UI components which present data that changes over time'' \cite[p.~1]{Hunt:2013}. Note that the example React code is not shorter than its pure JavaScript counterpart; there is still a fair amount of boilerplate in constructing the events to manage state. React sacrifices code simplicity for \emph{conceptual} simplicity, and therein lies its greatest strength.


\section{The Future is Reactive}
\vspace{-12pt}

\begin{quote}
	\singlespacing
	\emph{In which it is suggested that the functional and asynchronous paradigms uncovered by React will drive web design for the foreseeable future.}
\end{quote}

Thus far it has been shown that Angular and React were created to solve the same problems, but do so in different ways and with different objectives. What does this difference mean for the developer? Perhaps most succinctly, it can be said that Angular exists to aid the developer as much as possible in the \emph{present}, whereas React is a preparation for the \emph{future}.

Indeed, the design of React demonstrates considerable foresight. In nearly all software applications -- ranging from websites to video games to embedded systems -- the UI is nothing more than an alternate representation of the application data model, albeit much more complex and difficult to build. For one thing, there is no way to reliably test UI behavior short of human verification (does it ``look good''?) \cite{Hunt:2014}. Even more challenging, a typical UI is inherently asynchronous \footnote{This idea has gained quite a bit of traction in recent years, appearing under such guises as ``event-driven programming'' and the ``evented model.'' For a wealth of information on the subject in the realm of JavaScript, search for (in order of increasing sophistication): callbacks, promises, Functional Reactive Programming (FRP), and Communicating Sequential Processes (CSP).}, causing a multitude of problems for an industry that has been by convention imperative and sequential.

Crafting a UI is not the only challenge of building a modern web application, but it is certainly one of the toughest and most time-consuming. Applications will only get more complex. Interfaces will only get more dynamic. It is no coincidence that these are precisely the trends that React was meant to address by encouraging sensible application architecture and handling change over time. Thus, although React may not be the perfect solution for every web-based project, its design and core principles will likely heavily influence the web in coming years.


\section{Conclusion}
\vspace{-12pt}

\begin{quote}
	\singlespacing
	\emph{In which the points are tallied and a champion is crowned.}
\end{quote}

It may be tempting to conclude at this point that React is somehow ``better'' than Angular. However, this is not the case.

Asking a developer to judge the overall merits of a system is a common interview question. What is the best sorting algorithm? Which programming language is best for implementing a scalable backend? Is NoSQL better than SQL? \footnote{Spaghetti sort, Erlang, and no, respectively.} These are all, of course, trick questions to which the answer should be, ``It depends.'' It depends on the problem at hand. It depends on the collective experience of those involved. It depends on how much one is willing to invest in finding an answer.

In short, web frameworks -- like sorting algorithms, programming languages, and databases -- are just tools: which one is better suited to a project depends largely on the project itself. However, tools cannot be wielded if they are not first readily accessible. It is thus imperative that web developers expand their toolbelts; only then will they be ready to tackle the exciting, uncertain future of the web.


\newpage

\appendix
\section{Code Listings}
\label{app:code}

\subsection*{Pure JavaScript}

\lstinputlisting[style=html,caption={index.html},label={lst:pureHTML}]{../examples/pure-js/index.html}
\lstinputlisting[style=js,caption={app.js},label={lst:pureJS}]{../examples/pure-js/app.js}

\newpage

\subsection*{Angular}

\lstinputlisting[style=html,caption={index.html},label={lst:angularHTML}]{../examples/angular/index.html}
\lstinputlisting[style=js,caption={app.js},label={lst:angularJS}]{../examples/angular/app.js}

\newpage

\subsection*{React}

\lstinputlisting[style=html,caption={index.html},label={lst:reactHTML}]{../examples/react/index.html}
\lstinputlisting[style=js,caption={app.js},label={lst:reactJS}]{../examples/react/app.js}

\newpage

% BIBLIOGRAPHY (Note: We use a special version of IEEEtran that has sorting and annotations)
\begin{flushleft}
\begin{singlespace*}
	\bibliographystyle{./IEEEtran}
	\bibliography{angular-vs-react}
\end{singlespace*}
\end{flushleft}

\end{document}
